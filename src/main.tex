%%%%%%%%%%%%%%%%%%%%%%%%%%%%%%%%%%%%%%%%%
% fphw Assignment
% LaTeX Template
% Version 1.0 (27/04/2019)
%
% This template originates from:
% https://www.LaTeXTemplates.com
%
% Authors:
% Class by Felipe Portales-Oliva (f.portales.oliva@gmail.com) with template 
% content and modifications by Vel (vel@LaTeXTemplates.com)
%
% Template (this file) License:
% CC BY-NC-SA 3.0 (http://creativecommons.org/licenses/by-nc-sa/3.0/)
%
% Changed by Matthew for homework.
% 非商业使用
%%%%%%%%%%%%%%%%%%%%%%%%%%%%%%%%%%%%%%%%%

%----------------------------------------------------------------------------------------
%	PACKAGES AND OTHER DOCUMENT CONFIGURATIONS
%----------------------------------------------------------------------------------------

\documentclass[
	12pt, % Default font size, values between 10pt-12pt are allowed
%	UTF8
%	%letterpaper, % Uncomment for US letter paper size
	%spanish, % Uncomment for Spanish
]{fphw} %fphw
%\documentclass[12pt, UTF8]{ctexart}  %使用中文版的article文档类型排版,并选择UTF8编码格式

% Template-specific packages
\usepackage[UTF8]{ctex}	%使用ctex宏包,使得文中可以写中文
\usepackage[utf8]{inputenc} % Required for inputting international characters
\usepackage[T1]{fontenc} % Output font encoding for international characters
\usepackage{mathpazo} % Use the Palatino font

\usepackage{graphicx} % Required for including images
\usepackage{subfigure} %子图

\usepackage{booktabs} % Required for better horizontal rules in tables
\usepackage{listings} % Required for insertion of code
\usepackage{enumerate} % To modify the enumerate environment

%使得图片固定在某位置
\usepackage{float} 

%代码显示的包
\usepackage{listings}
\usepackage{xcolor}

%注释用
\usepackage{comment}
%----------------------------------------------
%配置代码显示格式-掌握minted之前的替代品
%----------------------------------------------
\definecolor{codegreen}{rgb}{0,0.6,0}
\definecolor{codegray}{rgb}{0.5,0.5,0.5}
\definecolor{codepurple}{rgb}{0.58,0,0.82}
\definecolor{backcolour}{rgb}{0.95,0.95,0.92}

\lstdefinestyle{mystyle}{
	backgroundcolor=\color{backcolour},   
	commentstyle=\color{codegreen},
	keywordstyle=\color{magenta},
	numberstyle=\tiny\color{codegray},
	stringstyle=\color{codepurple},
	basicstyle=\ttfamily\footnotesize,
	breakatwhitespace=false,         
	breaklines=true,                 
	captionpos=b,                    
	keepspaces=true,                 
	numbers=left,                    
	numbersep=5pt,                  
	showspaces=false,                
	showstringspaces=false,
	showtabs=false,                  
	tabsize=2
}

\lstset{style=mystyle}
%----------------------------------------------------------------------------------------


%----------------------------------------------------------------------------------------
%	ASSIGNMENT INFORMATION
%----------------------------------------------------------------------------------------

\title{第四章作业} % Assignment title

\author{Matthew} % Student name

\date{March 18th, 2077} % Due date
%\date{\today}

\institute{个人作业 \\ 视觉SLAM} % Institute or school name

\class{视觉SLAM理论与实践} % Course or class name

\professor{高翔} % Professor or teacher in charge of the assignment

%----------------------------------------------------------------------------------------

\begin{document}

\maketitle % Output the assignment title, created automatically using the information in the custom commands above

%----------------------------------------------------------------------------------------
%	ASSIGNMENT CONTENT
%----------------------------------------------------------------------------------------

\section*{2.图像去畸变}

\begin{problem}
	请根据所给文件中内容,完成对该图像的去畸变操作。
\end{problem}
%\begin{center}
%	\includegraphics[width=0.5\columnwidth]{pic1.png} % Example image
%\end{center}
%pic
\begin{comment}
 \begin{figure}[h]
	\centering
	\includegraphics[width=0.5\columnwidth]{pic1.png} % Example image
	\caption{测试图像}
	\label{test_pic}
\end{figure}
\end{comment}
%------------------------------------------------

\subsection*{Answer}

基本思路是...。添加的代码部分如下。
\begin{lstlisting}[language=C++, caption=题2所添代码]
	double x = code...;	//变换...	
\end{lstlisting}

得到的...。
\begin{figure}[ht]
	\centering
	\subfigure[测试图像]{               %小图题的名称
		\includegraphics[width=7cm]{pic1.png}}
	\hspace{0in}
	\subfigure[去畸变后图像]{
		\includegraphics[width=7cm]{pic1.png}}
	\caption{去畸变前后对比}
\end{figure}

%----------------------------------------------------------------------------------------
\clearpage
\section*{3.双目视差的使用}

\begin{problem}
	请根据给定参数,计算相机数据对应的点云,并显示到Pangolin 中。程序请参考code/disparity.cpp 文件。
	
	\medskip
	
%	\begin{enumerate}[(\itshape a\normalfont)] % Sub-questions styled as italic letters
%		\item Suppose ``chuck" implies throwing.
%		\item Suppose ``chuck" implies vomiting.
%	\end{enumerate}
\end{problem}

%------------------------------------------------

\subsection*{Answer}

在程序中添加上以下几行代码即可得到图2结果。

\begin{lstlisting}[language=C++, caption=题3所添代码]
	point[2] = xxxx;	
\end{lstlisting}

回答...

\begin{figure}[h]
	\centering
	\includegraphics[width=0.9\columnwidth]{pic1.png} % Example image
	\caption{运行结果}
	%\label{test_pic}
\end{figure}

\begin{comment}
\begin{enumerate}[(\itshape a\normalfont)] % Sub-questions styled as italic letters
	\item According to the Associated Press (1988), a New York Fish and Wildlife technician named Richard Thomas calculated the volume of dirt in a typical 25--30 foot (7.6--9.1 m) long woodchuck burrow and had determined that if the woodchuck had moved an equivalent volume of wood, it could move ``about \textbf{700 pounds (320 kg)} on a good day, with the wind at his back".
    
	\item A woodchuck can ingest 361.92 cm\textsuperscript{3} (22.09 cu in) of wood per day. Assuming immediate expulsion on ingestion with a 5\% retainment rate, a woodchuck could chuck \textbf{343.82 cm\textsuperscript{3}} of wood per day.
\end{enumerate}
\end{comment}
%----------------------------------------------------------------------------------------
\clearpage
\section*{4.矩阵运算微分}

\begin{problem}
	设变量为$x\in R^N$,那么:
	\begin{enumerate}
		\item 矩阵$A\in R^{N\times N}$,那么d(Ax)/dx 是什么?
		\item 矩阵$A\in R^{N\times N}$,那么d(xTAx)/dx 是什么?
		\item 证明 $xTAx = tr(AxxT)$
	\end{enumerate}
	\medskip
\end{problem}

%------------------------------------------------

\subsection*{Answer} 

\begin{enumerate}
	\item 本题采用的是原始定义证明法则。
	\begin{figure}[h]
		\centering
		\includegraphics[width=1.0\columnwidth]{pic1.png} % Example image
		%\caption{运行结果}
		%\label{test_pic}
	\end{figure}
	\item 第一问使用...查阅了相关资料\cite{matrix},了解到...方法\footnote{详见文献一p277。},下面使用...。
	\begin{figure}[H]
		\centering
		\includegraphics[width=1.0\columnwidth]{pic1.png} %  image
	\end{figure}
	\item 证明如下
	\begin{figure}[H]
		\centering
		\includegraphics[width=1.0\columnwidth]{pic1.png} %  image
	\end{figure}
\end{enumerate}

%----------------------------------------------------------------------------------------
\clearpage
\section*{5.高斯牛顿法的曲线拟合实验 (bonus marks)}

\begin{problem}
	现在请你书写Gauss-Newton的程序以解决此问题。程序框架见code/gaussnewton.cpp,请填写程序内容以完成作业
	
\begin{comment}
	\bigskip
    
	\begin{center}
		\begin{tabular}{l l l}
			\toprule
			\textit{Per 50g} & Pork & Soy \\
			\midrule
			Energy & 760kJ & 538kJ\\
			Protein & 7.0g & 9.3g\\
			Carbohydrate & 0.0g & 4.9g\\
			Fat & 16.8g & 9.1g\\
			Sodium & 0.4g & 0.4g\\
			Fibre & 0.0g & 1.4g\\
			\bottomrule
		\end{tabular}
	\end{center}
	
	\medskip
\end{comment}
\end{problem}

%------------------------------------------------

\subsection*{Answer}

所添加的第一处代码如下所示...

\begin{lstlisting}[language=C++, caption=题5所添代码]
	double code...;   // ...
\end{lstlisting}

在求解...
\begin{lstlisting}[language=C++, caption=题5所添代码] 
	Vector3d dx = code...;
\end{lstlisting}
回答...。
\begin{figure}[h]
	\centering
	\includegraphics[width=0.8\columnwidth]{pic1.png} % Example image
	\caption{运行结果}
	%\label{test_pic}
\end{figure}

%----------------------------------------------------------------------------------------
\clearpage
\section*{6.* 批量最大似然估计}

\begin{problem}
	\begin{comment}
	\lstinputlisting[
		caption=Luftballons Perl Script, % Caption above the listing
		label=lst:luftballons, % Label for referencing this listing
		language=Perl, % Use Perl functions/syntax highlighting
		frame=single, % Frame around the code listing
		showstringspaces=false, % Don't put marks in string spaces
		numbers=left, % Line numbers on left
		numberstyle=\tiny, % Line numbers styling
	]{luftballons.pl}
\end{comment}
	\begin{enumerate}
		\item 可以定义矩阵H,使得批量误差为$e=z-Hx$。请给出此处H 的具体形式。
		\item 请给出此问题下W 的具体取值。
		\item 假设所有噪声相互无关,该问题存在唯一的解吗?若有,唯一解是什么?若没有,说明理由。
	\end{enumerate}

\end{problem}

%------------------------------------------------

\subsection*{Answer}

\begin{enumerate}
	\item 证明如下:
	\begin{figure}[H]
		\centering
		\includegraphics[width=1.0\columnwidth]{pic1.png} 
	\end{figure}
	\item 证明如下:
	\begin{figure}[H]
		\centering
		\includegraphics[width=1.0\columnwidth]{pic1.png} 
	\end{figure}
	\item 该问...。
\end{enumerate}

%----------------------------------------------------------------------------------------

%
\begin{thebibliography}{99}    %参考文献开始
	\bibitem{matrix}张贤达.矩阵分析与应用.清华大学出版社,2004.                    %参考文献1
	\bibitem{2}知乎."三步搞定矩阵求导".https:https://zhuanlan.zhihu.com/p/262751195.         %参考文献2
	\bibitem{3}CSDN."机器人所涉及的相关数学理论整理".https://blog.csdn.net/GFDYVJG/article/details/109121221.
	\bibitem{4}"Relationship between the Hessian and Covariance Matrix for Gaussian Random Variables".https://onlinelibrary.wiley.com/doi/pdf/10.1002/9780470824566.app1
\end{thebibliography}
\addcontentsline{toc}{section}{参考文献}


\end{document}
